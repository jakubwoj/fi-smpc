
\section{Introduction}

Secure multi-party Computation (SMPC) is a cryptographic method to perform joint calculations by multiple parties without them getting to know each other's inputs.

% TODO Describe SMPC. Offline, online phase

The mathematical groundwork has been laid in the 90s, but SMPC was just considered theoretically for a long time.
Only in recent years, due to better performance and advancements in SMPC protocols, practical implementations have become feasible.

A few actively developed and maintained frameworks have been created.
Among them are e.g. Sharemind, FRESCO, Bristol SPDZ, ...
This paper looks at two of them in detail: FRESCO and Bristol SPDZ.

In section~\ref{sec:frameworks} each framework is described on it's own, while in section~\ref{sec:assessment} they are compared to each other.

\section{Considered SMPC frameworks}
\label{sec:frameworks}

The considered frameworks FRESCO and Bristol SPDZ are both active open-source SMPC solutions developed on GitHub.
And that are all the similarities, virtually everything else about them is different.
The frameworks use different tool-sets, follow different philosophies, and have been developed for different reasons.

\subsection{FRESCO}

Developed in Java.

Packaged with all dependencies in a JAR by Maven.

Meant as a abstraction allowing the same computations to be performed using different SMPC protocols.
Implements BGW, SPDZ.

% TODO Explain BGW, SPDZ

Computations to be written in Java.
Sequence of computation steps to be defined.
Compiled into single Java application.

\subsection{Bristol SPDZ}

Developed in C++, Python.

Dependencies have to be installed manually.

Implementation of the SPDZ protocol and an showcasing of the improved offline phase MASCOT\cite{KOS2016}.

% TODO Explain MASCOT

Computations to be written in Python.
Sequence of computation steps inferred automatically. % TODO check.
Multiple files generated.
Each phase has to run manually.


\section{Assessment of the frameworks}
\label{sec:assessment}

The frameworks are compared and assessed in respect to practicality and performance.

\subsection{Setup and methodology}

Plain VM. + goal: computing of ...

\subsection{Key (infrastructural) differences}

+ dependencies

\paragraph{Bristol SPDZ}

apt-get install g++

MPIR:
sudo apt-get install m4



\subsection{Usability}

Practicality

\subsubsection{FRESCO}

\subsubsection{Bristol SPDZ}


\subsection{Performance}

\subsubsection{FRESCO}

\subsubsection{Bristol SPDZ}


\section{Conclusion}
\label{sec:conclusion}


